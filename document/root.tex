\documentclass[11pt,a4paper]{article}
\usepackage[T1]{fontenc}
\usepackage{isabelle,isabellesym}

\usepackage{zed}

\renewcommand{\isasymZpfun}{\isamath{\pfun}}
\renewcommand{\isasymZpinj}{\isamath{\pinj}}

% further packages required for unusual symbols (see also
% isabellesym.sty), use only when needed

\usepackage{amssymb}
  %for \<leadsto>, \<box>, \<diamond>, \<sqsupset>, \<mho>, \<Join>,
  %\<lhd>, \<lesssim>, \<greatersim>, \<lessapprox>, \<greaterapprox>,
  %\<triangleq>, \<yen>, \<lozenge>

%\usepackage{eurosym}
  %for \<euro>

\usepackage{stmaryrd}
  %for \<Sqinter>

%\usepackage{eufrak}
  %for \<AA> ... \<ZZ>, \<aa> ... \<zz> (also included in amssymb)

\usepackage{textcomp}
  %for \<onequarter>, \<onehalf>, \<threequarters>, \<degree>, \<cent>,
  %\<currency>

% this should be the last package used
\usepackage{pdfsetup}

% urls in roman style, theory text in math-similar italics
\urlstyle{rm}
\isabellestyle{it}

% for uniform font size
%\renewcommand{\isastyle}{\isastyleminor}


\begin{document}

\title{Z Mathematical Toolkit in Isabelle/HOL}
\author{Simon Foster, Pedro Ribeiro and Frank Zeyda \\[.5ex] University of York, UK \\[2ex] \texttt{\small simon.foster@york.ac.uk}}
\maketitle

\begin{abstract}
\noindent The objective of this theory development is an implementation of the Z mathematical 
toolkit in Isabelle/HOL that is both efficient for proof and faithful to the standard.
We construct the Z metalanguage and type universe on top of HOL, and link this to
corresponding concrete types (finite functions, lists etc.) in Isabelle, to enable
efficient proof automation. We then utilise coercive subtyping and overloading to
support processing of Z-like expressions in Isabelle/HOL. We then use this to develop
the mathematical toolkit for sets, relations, functions, and sequences.
\end{abstract}

\tableofcontents
\newpage

% sane default for proof documents
\parindent 0pt\parskip 0.5ex

% generated text of all theories
\input{session}

% optional bibliography
\bibliographystyle{abbrv}
\bibliography{root}

\end{document}

%%% Local Variables:
%%% mode: latex
%%% TeX-master: t
%%% End:
